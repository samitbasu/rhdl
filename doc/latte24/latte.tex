\documentclass[sigplan,screen,sigconf]{acmart}
\usepackage{tcolorbox}
\usepackage{minted}
\settopmatter{printfolios=false,printacmref=false}
\bibliographystyle{ACM-Reference-Format}

\setcopyright{rightsretained}
\acmDOI{}
\acmISBN{}
\acmConference[LATTE '24]{3nd Workshop on Languages, Tools, and Techniques for Accelerator Design}{March 26, 2023}{Vancouver, BC, Canada}

\author{Samit Basu}
\affiliation{
  basu.samit@gmail.com
  \country{Fremont CA, USA}
}

\begin{document}

\title{Rust as a Hardware Description Language}

\begin{abstract}
Rust \cite{b9} makes an excellent language for hardware description.  A number of new HDLs are 
Rust-inspired in syntax \cite{b1},\cite{b4},
but RustHDL\cite{b6} is actually a framework for turning ordinary procedural Rust code into firmware.  The first attempt,
while successful had several shortcomings that are discussed in this paper.  The new framework, RHDL\cite{b10},
should address those shortcomings and significantly ease the use of Rust as an HDL.
\end{abstract}

\maketitle

\section{Introduction}

While the field of HDLs may be crowded, I propose the use of the Rust Programming Language (RPL)
as a hardware description language.  Beyond its status as most loved of the programming languages \cite{b0},
Rust has steadily been gaining traction as a serious language for systems programming, embedded 
software, and other mission critical applications.  The particular features of Rust that are 
relevant to hardware description include:
\begin{itemize}
\item Static typing and sane syntax, which can prevent many errors at compile time,
\item Functional programming features including pattern matching and iterators for testing.
\item A powerful macro system that allows for meta-programming.
\item Package management and a growing open-source ecosystem to ease collaboration and reuse.
\item Built-in test capabilities to minimize ``bench''-time.
\item Significant tooling and infrastructure support in the form of linters, analysis tools, etc.
\item Generics and const generics to allow for parameterized designs.
\end{itemize}

The RHDL framework and its predecessor RustHDL take advantage of all of these features to
create an environment for hardware description that is powerful, easy to use, extensible and open.  The end goal is to enable a wider class of engineers to develop high quality hardware by reusing their skills as Rust developers in the hardware domain.  This paper briefly describes the RustHDL approach to developing FPGA firmware using the RPL, and then identifies the observed shortcomings and how they may be addressed in the upcoming RHDL framework.  Finally, I touch upon some of the unsolved problems in using Rust for hardware description.  Note that I use the term ``hardware description'' here to mean FPGA firmware (or possibly ASIC designs), as Rust is already established as a language for embedded systems programming (which is also often referred to as ``firmware''). 
I think it's also important to point out that the appeal of using Rust as an HDL is it's familiarity to engineers coming from a C background.  The syntax is ``C-like'', and
while functional programming concepts are supported, the language is multi-paradigm.  Given that part of the objective of RustHDL and RHDL is to enable a wider class of engineers to develop hardware, this is a critical point.

\section{Syntax}
RustHDL is not a new language.  Instead it is a set of libraries and macros along with a 
subset of the Rust programming language that can be used to generate firmware.  The key
principle of RustHDL is:

\begin{tcolorbox}
RustHDL designs are valid Rust programs that can be compiled and run on a host computer
using the included event-based simulator.
\end{tcolorbox}

In this sense, RustHDL is embedded in the Rust language much as MyHDL is embedded in Python \cite{b3},
and Chisel is embedded in Scala \cite{b2}.  The immediate implication is that:
\begin{itemize}
  \item All RustHDL designs must pass the strenuous correctness checks of the Rust compiler \verb|rustc|.
  \item An entire class of potential bugs are eliminated, such as type mismatches, use-before-initialization,
  unassigned outputs, etc.
  \item Tools such as \verb|clippy| and \verb|rust-analyzer| can immediately be used to
  check, lint and analyze code with no additional investment.
  \item The Rust test framework can be used to test the designs directly.
\end{itemize}
Note that unlike MyHDL, RustHDL does not use a generator pattern and infer the required hardware.  Instead, the AST itself is transformed into the circuit description.  So, for example, the following AST fragment describes (behaviorally) a mux that selects between the current counter output \verb|counter.q| and \verb|counter.q+1|.  The description is not ``builder'' style, in which a MUX is explicitly instantiated.  The MUX is inferred from the imperative code.  This is also in contrast to a combinator style of hardware description, for example in \cite{b4b}, where the language used is Rust, but hardware description is functional.

\begin{minted}[fontsize=\footnotesize]{rust}
  // signal for false control signal --v
  self.counter.d.next = self.counter.q.val();
  //    v-- MUX control signal
  if self.enable.val() {
  // signal for true control signal  --v
    self.counter.d.next = self.counter.q.val() + 1;
  }
\end{minted}

The syntax should be fairly familiar to anyone comfortable with Rust (or C++).  The following is an example of a simple SPI master in RustHDL, generic over the transaction size \verb|N|, edited for brevity:

\begin{minted}[fontsize=\footnotesize]{rust}
  #[derive(LogicBlock)]
  pub struct SPIMaster<const N: usize> {
      pub clock: Signal<In, Clock>,
      pub data_outbound: Signal<In, Bits<N>>,
      pub start_send: Signal<In, Bit>,
      pub data_inbound: Signal<Out, Bits<N>>,
      pub wires: SPIWiresMaster,
      local_signal: Signal<Local, Bit>,
      state: DFF<SPIState>,
      cs_off: Constant<Bit>,
  }
\end{minted}  

The \verb|pub| keyword is used to denote the visibility of the signals.  Signals 
marked with a direction, and type.  Internal components such as flip-flops and strobes
are all included in the top level struct, which is initialized using normal Rust code.
The \verb|Local| signal represents a local variable used in the update function, but
not otherwise exposed.  As RustHDL had no type inference, it requires explicit allocation
and types for all local variables.  The member \verb|cs_off| (along with others omitted) is a constant constructed at runtime that encodes the SPI mode of the bus.  Finally, 
the \verb|SPIWiresMaster| is a struct that describes the interface to the actual SPI bus.
Interfaces (unlike structs in SystemVerilog, for example) include both input and output
signals, and can be used to ``connect'' complex components with a single line.  

Note that in this instance, \verb|state: DFF<SPIState>| is equivalent to a module instantiation.  The \verb|DFF| is a flip-flop, and \verb|SPIState| is a C-style enum that represents the state of of the controller.  By including it as a member of the struct, we request an instance of it be created in the generated design.  Thus, composition of 
modules is equivalent to composition of data structures.

RustHDL (but \emph{not} RHDL) supports bidirectional interface declarations which can 
be used to connect complex components together in a type-safe way.
As an example, an interface to an SDRAM chip with a \verb|D-bit| data bus and a 13 bit 
address bus is defined as:

\begin{minted}[fontsize=\footnotesize]{rust}
  #[derive(LogicInterface, Clone, Debug, Default)]
  #[join = "SDRAMDriver"]
  pub struct SDRAMDevice<const D: usize> {
      pub clk: Signal<In, Clock>,
      pub we_not: Signal<In, Bit>,
      pub cas_not: Signal<In, Bit>,
      pub ras_not: Signal<In, Bit>,
      pub address: Signal<In, Bits<13>>,
      pub write_data: Signal<In, Bits<D>>,
      pub read_data: Signal<Out, Bits<D>>,
      pub write_enable: Signal<In, Bit>,
  }
\end{minted}

and can be connected to the corresponding signals in another IP block with a 
single \verb|join| statement.  This significantly reduces the amount of error-prone wiring that must be done by code or graphically.  The \verb|join| statement is used inside of an \verb|update| function as the following demonstration:

\begin{minted}[fontsize=\footnotesize]{rust}
#[derive(LogicBlock)]
struct I2CControllerTest {
    clock: Signal<In, Clock>,
    controller: I2CController,
    target_1: I2CTestTarget,
    target_2: I2CTestTarget,
    test_bus: I2CTestBus<3>,
}

impl Logic for I2CControllerTest {
    #[hdl_gen]
    fn update(&mut self) {
        clock!(self, clock, controller, target_1, target_2);
        I2CBusDriver::join(&mut self.controller.i2c, 
          &mut self.test_bus.endpoints[0]);
        I2CBusDriver::join(&mut self.target_1.i2c,
          &mut self.test_bus.endpoints[1]);
        I2CBusDriver::join(&mut self.target_2.i2c, 
          &mut self.test_bus.endpoints[2]);
    }
}
\end{minted}

In this example, the controller, and 2 DUTs are connected to a bus.  Since all of the logic is simply connecting the interfaces together, it consists mainly of \verb|join| statements. 

Back to the SPI controller example, an update function calculates the next value of the 
signals (external and internal) based on the current state stored in the DFF \verb|state|,
which itself is a C-style enum.  Rust ensures that the state machine match/case is exhaustive. 

\section{Mental Model}
RustHDL attempts to build on HDLs like Lucid\cite{b11} to provide a more understandable mental model for how hardware works.  In an imperfect implementation, RustHDL defines a \verb|Signal| struct that has a read only endpoint \verb|x.val()| for signal \verb|x|, and a write endpoint \verb|x.next|.  

\begin{minted}[fontsize=\footnotesize]{rust}
// Design is parametric over N - the size of the counter
impl<const N: usize> Logic for Strobe<N> {
  #[hdl_gen]
  fn update(&mut self) {
    if self.enable.val() {
      self.counter.d.next = self.counter.q.val() + 1;
    }
    self.strobe.next = self.enable.val() & 
      (self.counter.q.val() == self.threshold.val());
    if self.strobe.val() {
      self.counter.d.next = 1.into();
    }
  }
} 
\end{minted}

Rust lacks write-only semantics, so the framework checks for read-before-write on the \verb|x.next| endpoint.  Using this nomenclature, the idea of non-blocking assignments is replaced with a conditional model - i.e., given the current value in the set of signals, what next value do I want them to take?  This mental model is coupled with analysis passes that look (with the aid of Yosys\cite{b12} in RustHDL) for latch inferences due to missing assignments and other such issues.

The mental model of RustHDL is not ideal (and is replaced in RHDL).  However, the main advantage it has is that it is very ``normal'' looking.  A signal's \verb|.next| endpoint can be written to as many times as desired inside of an \verb|update| function.  Only the last value it takes will matter when the function completes.  In essence, the last successful write to a signal ``wins'', where success may be conditional (in this case, for example, the value of \verb|self.counter.d.next| depends on the value of 
\verb|self.enable.val()|).  For synchronous logic this concept can be expressed as:

\begin{tcolorbox}
The current value of the signal is accessible via \verb|.val()|, and the value that 
signal will take on the next clock cycle will be decided by the last assignment
to \verb|.next|.  I.e., in the next clock cycle \verb| next -> val|.
\end{tcolorbox}

Again, this mental model has been replaced and simplified in RHDL, but the 
core concept is still similar. Local variables can be both written to and read, 
functioning as scratchpads, as long as a write precedes any subsequent reads or writes.  Again, all of these (somewhat complex) rules are replaced in RHDL with a much simpler model.


\section{Simulation}
Testing of designs in RustHDL does not require the use of third party tools or tooling.  
Tests utilize a built-in event-based simulation engine that can simulate any RustHDL design. Black box IP cores can be simulated by providing Rust equivalents of the hardware descriptions. The simplest example is something like a block RAM, which can be trivially instantiated in Verilog, but requires a behavioral model in RustHDL.  In RustHDL that behavioral model is written in Rust, and can be substituted into the simulation environment.  Other black box IP cores can be equivalently simulated in Rust.  Note that because RustHDL supports combinatorial connections across modules that the simulator iterates until it reaches a fixed point.  The iterations will terminate with an error if some upper limit is reached.

Speed is a critical factor in simulation.  RustHDL is a reasonably fast simulator, 
and the Rust test framework is inherently parallel, and can run multiple tests in parallel. Using system calls/shell-outs, the entire synthesis and bitstream generation process can be handled inside the Rust ecosystem.  A direct comparison with Verilator proved difficult as Verilator rejected the Verilog generated by RustHDL (possibly due to the presence of asynchronous logic in the design).  A fair comparison is a target for RHDL, which supports multi-threaded simulation, and should be even faster than RustHDL.

\section{Reuse}
Hardware designs in RustHDL are simply \verb|struct|s, and are composed of other 
hardware components via composition.  This allows for easy reuse of components, the
construction of complex designs out of simpler, smaller components, combined with sane rules of scoping and encapsulation.  Furthermore, each of the sub-components can be tested in isolation, and then tested after composition in the larger design. 

Rust is a very composable language, and \verb|crates.io| provides a natural mechanism
for sharing and reusing components.  As an example, in RustHDL, handling of hardware specific details (such as synthesis tools and constraints files for specific FPGAs and boards) is handled through a \emph{board support package}.  This is simple a library that provides the defaults, pin-outs, and other mapping information needed to generate a bitstream for a given piece of hardware.  As an open-ended and potentially unbounded problem, the BSP can be published as a \verb|crate| (package) in the Rust ecosystem by contributors \cite{b7}.  This decentralizes control over one of the more challenging parts of maintaining support for a bewildering array of devices. 

Meta-programming is supported in RustHDL, but to a somewhat limited extent.  Most of the meta-programming is provided by macros (procedural and declarative) that generate the necessary code.   FIFOs that require various generic parameters can be instantiated via a simple macro.  And interfaces use macros to describe mating interfaces with signals of opposite direction.  

\section{Shortcomings and the Future}
RustHDL has been used for non-trivial commercial firmware development and is deployed.  It has also seen some level of interest and adoption from the open source community.  Feedback from early users lead to the following list of shortcomings:

\begin{itemize}
  \item The subset of Rust supported by RustHDL (which is the subset of the language that can be   directly translated into Verilog) is too small to write ``natural'' Rust code. 
  \item RustHDL does not support Algebraic Data Types (data-carrying enums).
  \item Local variables and type inference are critical to writing clean and
  idiomatic Rust code.  
  \item Composition of functions/behavior is not possible. 
  \item Writing test-benches required an understanding of the simulator mechanics.
  \item Backends are needed for more than just Verilog.
\end{itemize}

Solving all of these problems essentially necessitated a rewrite of RustHDL.  The new framework, called RHDL (Rust Hardware Description Language) is currently under development.  The primary technical difference to RustHDL is the introduction of an auxiliary compiler into the processing. This compiler works in tandem with \verb|rustc| to convert an AST of the code into a RTL-like HDL, and then transform and optimize that representation into a form that can be synthesized.  The  compiler is key to support of things like early returns, match and if expressions (as opposed to statements), and other Rust-isms that are not common in HDLs, but are common in Rust.  The compiler also provides
ADT support with control over the layout of the data, and easy composition of data types into structs of arbitrary complexity.  

\section{Conclusions}
I believe Rust is a promising basis for a hardware description language.  It offers many powerful tools that can be utilized to build composable, reusable and correct hardware designs.  The RustHDL framework was a first step in this direction, and the in-development RHDL framework promises to address many of the shortcomings of the first attempt. I look forward to presenting more details on RHDL as it evolves.

\begin{thebibliography}{00}
  \bibitem{b9} ``Rust - A language empowering everyone to build reliable and efficient software'', \url{https://rust-lang.org} (Accessed Feb 1, 2024).
  \bibitem{b1} F. Skarman and O. Gustafsson, ``Spade: An Expression-Based HDL With Pipelines'', Open Source Design Automation Conference, 2023.
  \bibitem{b4} ``XLS: Accelerated HW Synthesis'', \url{https://google.github.io/xls/} (Accessed Feb 1, 2024).
  \bibitem{b4b} Sungsoo Han, Minseong Jang, and Jeehoon Kang, ``ShakeFlow: Functional Hardware Description with Latency-Insensitive Interface Combinators'', ASPLOS 2023.
  \bibitem{b6} ``RustHDL - Write FPGA Firmware using Rust!'', \url{https://rust-hdl.org/} (Accessed Feb 1, 2024).
  \bibitem{b10} ``RHDL - Rust Hardware Description Language'', \url{https://github.com/samitbasu/rhdl} (Accessed Feb 1, 2024).
  \bibitem{b0} ``Stack Overflow Developer Survey 2023'', \url{https://insights.stackoverflow.com/survey/2023} (Accessed Feb 1, 2024).
  \bibitem{b3} ``MyHDL - From Python to Silicon!'', \url{https://www.myhdl.org/} (Accessed Feb 1, 2024).
  \bibitem{b2} ``Chisel - Software-defined hardware'', \url{https://www.chisel-lang.org/} (Accessed Feb 1, 2024).
  \bibitem{b11} J. Rajewski, ``Lucid - FPGA Tutorials'', \url{https://alchitry.com/lucid/} (Accessed Feb 18, 2024).
  \bibitem{b12} C. Wolf, ``Yosys Open SYnthesis Suite'', \url{https://yosyshq.net/yosys/} (Accessed Feb 18, 2024).
  \bibitem{b7} ``rust-hdl-bsp-step-mxo2-lpc - rust-hdl board support package for STEP-MXO2-LPC'', \url{https://crates.io/crates/rust-hdl-bsp-step-mxo2-lpc} (Accessed Feb 4, 2024).
\end{thebibliography}
\end{document}
